\pgfplotsset{
    integral segments/.code={\pgfmathsetmacro\integralsegments{#1}},
    integral segments=10,
    integral/.style args={#1:#2}{
        ybar interval,
        domain=#1+((#2-#1)/\integralsegments)/2:#2+((#2-#1)/\integralsegments)/2,
        samples=\integralsegments+1,
        x filter/.code=\pgfmathparse{\pgfmathresult-((#2-#1)/\integralsegments)/2}
    }
}

%% https://math.dartmouth.edu/opencalc2/cole/lecture8.pdf
%%\begin{tikzpicture}[/pgf/declare function={f=0.1*x^2+1;}]
\begin{tikzpicture}[/pgf/declare function={f=3*x^3-6*x^2+2*x-1;}]
\begin{axis}[
    domain=0:10,
    samples=100,
    axis lines=middle,
    xticklabel=$t_{\pgfmathprintnumber{\tick}}$
]
\addplot [
    red,
    fill=red!50,
    integral=0:10
] {f};
\addplot [thick] {f};
\end{axis}
\end{tikzpicture}
